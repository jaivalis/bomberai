%\begin{document} 
For this project we chose to implement two types of artificial intelligent agents for the game Bomberman.

%\subsection{A Brief Introduction}
The game Bomberman (also known as Dynablaster) is a strategic, maze-based, 2D
video game franchise originally developed by Hudson Soft. It was originally
published in 1983 and until now over 80 games have featured
Bomberman\cite{bomberman2013}.
The goal of the game is to complete levels by strategically placing bombs in order to eliminate enemy players and demolish obstacles. 
Bombs are the only weapon at the players disposal and are a key aspect of the
game. The bombs will explode a brief amount of time after they have been placed
and will kill any player that is within their reach and demolish up to one
obstacle in each of the four directions they affect. 

Power ups are an essential part in Bomberman as they increase the reach of the
fire of the bombs, the speed of the player or how many bombs one player can have
placed on the map at the same time. One can imagine that these power ups are
advantageous as they allow the player to gain more powers, speed to or from the
enemy and surround enemies entirely with bombs. In our version of Bomberman
there are two upgrades available: lay one additional bomb simultaneously, or
increase the range of a bomb with one (in each direction). Both upgrades
can be attained several times, stacking their effects.

To investigate the effect of power ups we have crafted two artificial intelligent agents which both employ different strategies in regards to upgrades. One AI employs a strategy to get to and kill an enemy player as fast as possible, we shall call this AI `Kill-First-AI'. The second AI first attempts to gain as much upgrades as possible before engaging the other players, we shall call this AI `Upgrade-First-AI'.
