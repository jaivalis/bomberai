%\begin{document}
Can we now honestly advocate one type of strategy over another when you play
Bomberman at home with you friends? Obviously, no.
First of all, the game in which we researched the strategies differs in some
ways from regular Bomberman games:
\begin{itemize}
\item Things might be different with different upgrades. Upgrades which increase
the speed of the player or which allow the player to `kick' the bombs to different
locations than they were originally laid down can be more beneficial. This could
mean that getting such upgrades before going in for the kill becomes essential
to succes, making the `Upgrade-First-AI' better.
%\item `Normal' Bomberman works differently in regards to collision detection,
%which allows players passing through each other.
\item Different kind of maps might require different kind of strategies. In
original Bomberman games there are additional map features, for example
teleports or tunnels. This might influence the effect certain strategies have.
\end{itemize}

In regards to the AI for this game we can conclude that it is better to go in
for the kill as soon as possible, but there are some factors that we should
not rule out yet:
\begin{itemize}
\item Tweaks in low level behaviour might favor one type of AI more that the
other. For example, planning to destroy as much block as possible
(Upgrade-First-AI) is different from destroying those blocks which opens up
a path to the opponent as quickly as possible (Kill-First-AI). One of the
methods we implemented might work better, favoring the low level behavior
of the bots rather than the grand strategy we try to evaluate.
\item Human players might not be able to utilize all the upgrades. For example,
the AI might be able to plan and actually drop 20 bombs on the map, but a human player
would not be able to move quickly enough and lay the bombs correctly, resulting
in perhaps an accidental suicide. This could mean that the upgrading to the max
could be a less viable strategy for human players.
\end{itemize}
